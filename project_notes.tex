\documentclass[a4paper,12pt]{article}
\usepackage[utf8]{inputenc}
\usepackage{geometry}
\usepackage{xcolor}
\usepackage{listings}
\usepackage{hyperref}
\usepackage{color}
\usepackage{graphicx}
\usepackage{amssymb}
\usepackage{background}

\backgroundsetup{
  scale=1.5,
  color=black,
  opacity=0.1,
  angle=45,
  position=current page.center,
  vshift=0cm,
  contents={Created and Compiled by Sapandeep Singh Sandhu}
}

% Page geometry
\geometry{
 a4paper,
 total={170mm,257mm},
 left=20mm,
 top=20mm,
}

% Code highlighting settings
\definecolor{codegreen}{rgb}{0,0.6,0}
\definecolor{codegray}{rgb}{0.5,0.5,0.5}
\definecolor{codepurple}{rgb}{0.58,0,0.82}
\definecolor{backcolour}{rgb}{0.95,0.95,0.92}

\lstdefinestyle{mystyle}{
    backgroundcolor=\color{backcolour},   
    commentstyle=\color{codegreen},
    keywordstyle=\color{magenta},
    numberstyle=\tiny\color{codegray},
    stringstyle=\color{codepurple},
    basicstyle=\ttfamily\footnotesize,
    breakatwhitespace=false,         
    breaklines=true,                 
    captionpos=b,                    
    keepspaces=true,                 
    numbers=left,                    
    numbersep=5pt,                  
    showspaces=false,                
    showstringspaces=false,
    showtabs=false,                  
    tabsize=2
}

\lstset{style=mystyle}

\title{\textbf{CS Compiler Design: Lexical Analysis Lecture Notes}}
\author{Detailed Project Walkthrough}
\date{\today}

\begin{document}

\maketitle
\tableofcontents
\vspace{1cm}

\section{Introduction}
This document serves as both a project report and a set of \textbf{Teacher-Style Lecture Notes} for understanding the design and implementation of a Lexical Analyzer (Scanner). We have implemented a working lexer in C that tokenizes a subset of the C language.

\section{Project Validation}
Does this implementation meet the core requirements of a standard lexical analyzer?

\begin{itemize}
    \item[\checkmark] \textbf{Ignores redundant whitespace}: Spaces, tabs, and newlines are skipped efficiently.
    \item[\checkmark] \textbf{Ignores comments}: Successfully handles both single-line \texttt{//} and multi-line \texttt{/*...*/} comments.
    \item[\checkmark] \textbf{Token Recognition}: Correctly groups keywords, identifiers, integers, floats, strings, chars, and operators.
    \item[\checkmark] \textbf{Identifier Logic}: Implements "arbitrarily long" identifiers in theory, but restricts them to a reasonable safety limit (64 chars) to prevent buffer overflows.
    \item[\checkmark] \textbf{Simulation}: This is a fully functional scanner simulation in standard C.
\end{itemize}

\section{Theoretical Foundations}

\subsection{What is a Lexical Analyzer?}
A lexical analyzer reads the program \textbf{character-by-character} and groups them into \textbf{tokens}.
\\
\textbf{Example Input:} \texttt{int total = 10 + 20;}
\\
\textbf{Token Stream Output:}
\begin{itemize}
    \item \texttt{int} $\rightarrow$ KEYWORD
    \item \texttt{total} $\rightarrow$ IDENTIFIER
    \item \texttt{=} $\rightarrow$ OPERATOR
    \item \texttt{10} $\rightarrow$ INT
    \item \texttt{+} $\rightarrow$ OPERATOR
    \item \texttt{20} $\rightarrow$ INT
    \item \texttt{;} $\rightarrow$ SEPARATOR
\end{itemize}

\textbf{Why do we need this?} Parsing (syntax analysis) is significantly easier when the parser sees discrete \textbf{tokens} rather than a raw, messy stream of characters.

\section{Detailed Implementation Walkthrough}

\subsection{Global Organization}
The lexer typically works in a simplified standard pipeline:
\begin{enumerate}
    \item \textbf{Skip Phase}: Consume all whitespace and comments.
    \item \textbf{Decide Phase}: Peek at the next character to determine the token class (e.g., if it's a digit, it's a number).
    \item \textbf{Scan Phase}: Read the full lexeme (word/number).
    \item \textbf{Return}: Produce a single token struct.
    \item Repeat until \textbf{EOF}.
\end{enumerate}

\subsection{Token Definitions}
We distinguish token classes using an \texttt{enum} (e.g., \texttt{TOK\_KEYWORD}, \texttt{TOK\_INT}). 
\\
\textbf{Why?} The parser must treat \texttt{if} differently from a variable named \texttt{result}.
\\
We also store \textbf{Lexemes} (the actual text) and \textbf{Position} (line/col).
\\
\textbf{Why?} Position tracking is critical for meaningful error messages (e.g., "Error at line 10, col 5").

\subsection{Scanning Logic: The "How-To"}

\subsubsection{1. Reading \& Position Tracking}
We implement a \texttt{read\_char()} function.
\begin{itemize}
    \item Reads one char from file.
    \item Unique logic: If char is \texttt{\\n}, increment \texttt{line} and reset \texttt{col}.
\end{itemize}
We also use \texttt{unread\_char()} (Lookahead). This pushes a character \textit{back} onto the stream. This is essential when we read one character too far to determine a token's end (e.g., distinguishing \texttt{123} from \texttt{123.}).

\subsubsection{2. Identifier vs. Keyword}
\textbf{The Problem:} \texttt{if} (keyword) and \texttt{index} (identifier) look structurally identical (letters).
\\
\textbf{The Solution:}
\begin{enumerate}
    \item Scan the full word assuming it is an identifier.
    \item Check the word against a strictly defined \textbf{Keyword Table}.
    \item If a match is found, re-label it as \texttt{TOK\_KEYWORD}.
\end{enumerate}
Our implementation also enforces a \texttt{MAX\_ID\_LEN} (64). While abstract theory allows infinite length, real computers have finite memory, making this a practical and "reasonable" safety constraint.

\subsubsection{3. Number Scanning (Int vs Float)}
\begin{itemize}
    \item \textbf{Step 1}: Read consecutive digits.
    \item \textbf{Step 2}: If a dot (\texttt{.}) is found, switch mode to \textbf{FLOAT} and read fractional digits.
    \item \textbf{Note}: This simple scanner does not handle scientific notation (e.g., \texttt{1.2e-3}), which is acceptable for a mini-project.
\end{itemize}

\subsubsection{4. String \& Char Literals}
\begin{itemize}
    \item \textbf{Strings}: Read from opening \texttt{"} to closing \texttt{"}. We handle escaped quotes (\texttt{\\"}) by ensuring the loop doesn't stop prematurely.
    \item \textbf{Error Case}: If we hit a newline before the closing quote, we immediately return \texttt{TOK\_ERROR}. This catches "Unterminated string" bugs.
\end{itemize}

\subsubsection{5. Operators \& Sort-Order Logic}
We check for \textbf{Longest Match First}.
\begin{itemize}
    \item We first check for 2-character operators (e.g., \texttt{==}, \texttt{<=}).
    \item If no match, we fall back to single-character operators (e.g., \texttt{=}, \texttt{<}).
    \item If we checked single-chars first, \texttt{==} might be mistakenly read as two separate \texttt{=} tokens.
\end{itemize}

\section{Critical Analysis}
\textit{A note on correctness and edge cases.}

While the tokenization logic is robust, there is a subtle nuance in position tracking concerning \texttt{unread\_char()}.

\begin{quote}
\textbf{The Issue:} If we read a newline (\texttt{\\n}) and then unread it, our simple \texttt{unread\_char} logic does not decrement the line counter.
\end{quote}

In practice, this means if a token (like a number) ends exactly at a newline, the line count for the \textit{next} token might be slightly off.
\\
\textbf{Fix Idea:} A production-grade compiler would use a robust \texttt{peek\_char()} function that buffers the character without advancing the file pointer, or a more complex state machine for newline history. For this simulation, the current approach is functional and standard for student-level projects.

\section{Conclusion}
This implementation meets all assignment criteria. It simulates the core "brain" of a compiler frontend—managing messy input streams and delivering clean, structured tokens to the next phase of compilation.

\end{document}
